Em meio ao contínuo crescimento urbano e à expansão das redes viárias, a eficiente gestão do tráfego torna-se imprescindível para garantir a fluidez e a segurança no transporte urbano.
Nesse contexto, os cruzamentos viários desempenham um papel crucial, representando pontos críticos onde a coordenação entre os semáforos se torna essencial para otimizar o fluxo de veículos.
Este artigo se concentra na modelagem de um cruzamento composto por quatro semáforos, utilizando Redes Coloridas de Petri.

As Redes Coloridas de Petri, uma extensão das tradicionais Redes de Petri, emergem como uma ferramenta poderosa para representar sistemas dinâmicos e distribuídos.
Ao aplicar esta metodologia à modelagem de cruzamentos viários, buscamos proporcionar uma compreensão mais aprofundada da interação entre os semáforos, concentrando-nos exclusivamente na comunicação entre esses elementos-chave.
A simplificação deliberada do escopo do estudo visa oferecer uma visão clara e concisa, permitindo uma análise mais eficaz da dinâmica de controle de tráfego no cruzamento.

Exploraremos, ao longo deste artigo, a aplicação prática das Redes Coloridas de Petri na modelagem de sistemas de tráfego, observando o controle de quatro semáforos em um cruzamento.
Ao fazer isso, almejamos contribuir para o avanço da compreensão teórica e prática dos sistemas de controle de tráfego, abrindo caminho para estratégias mais eficientes e adaptáveis na gestão do tráfego urbano.