A modelagem de sistemas complexos, como cruzamentos viários, requerem abordagens analíticas robustas que possam capturar de modo eficiente as dinâmicas envolvidas.
As Redes de Petri, introduzidas por Carl Adam Petri na década de 60, fornecem uma estrutura gráfica poderosa para a representação e análise de sistemas dinâmicos e concorrentes.

Uma Rede de Petri é composta por lugares, transições, tokens e arcos direcionados.
Os lugares representam estados do sistema, as transições indicam eventos ou ações, os tokens denotam a presença de recursos ou condições, e os arcos modelam as relações e transições permitidas~\cite{petri-nets-paper}.

As Redes Coloridas de Petri são uma extensão desta teoria que incorporam atributos coloridos para enriquecer a representação de informações e estados do sistema.
A adição de cores permite uma modelagem mais expressiva, possibilitando a análise de sistemas com características mais intrínsecas e variáveis~\cite{colored-petri-nets-book}.
A utilização de Redes Coloridas de Petri é relevante em contextos onde a representação precisa de múltiplos estados do sistema é crucial, como é o caso da dinâmica de semáforos em um cruzamento.

Um dos pilares fundamentais das Redes Coloridas de Petri é a capacidade de representar eventos concorrentes e a evolução temporal de um sistema.
Este formalismo fornece uma base sólida para a análise de sistemas dinâmicos complexos, incluindo aqueles encontrados em aplicações de controle de tráfego~\cite{petri-nets-applications-article}.

Ao integrar os princípios fundamentais das Redes Coloridas de Petri com a dinâmica específica de cruzamentos viários que envolvem quatro semáforos, este estudo visa contribuir para o avanço da teoria e prática no campo do controle de tráfego, promovendo estratégias mais eficazes e adaptáveis na gestão do tráfego urbano.