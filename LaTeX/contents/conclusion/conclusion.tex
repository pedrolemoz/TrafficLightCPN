Este artigo concentrou-se na modelagem de um cruzamento de semáforos por meio de uma Rede Coloridas de Petri, explorando a capacidade intrínseca dessas redes para representar sistemas dinâmicos e complexos.
Os resultados obtidos ao executar a Rede Colorida de Petri por meio da ferramenta CPN IDE forneceram insights significativos sobre o comportamento do sistema em diferentes estados.

No estado inicial, a rede apresentou todos os semáforos no estado fechado, indicado pela cor vermelha.
Esta configuração inicial permitiu a ativação de transições responsáveis por mover fichas dos lugares vermelhos para os verdes, com cada lugar vermelho inicialmente contendo duas fichas.
Esse estado inicial demonstrou a flexibilidade e adaptabilidade da Rede de Petri, possibilitando a execução arbitrária de transições.

Ao considerar o disparo de uma transição em um semáforo específico, ilustrado pela Figura~\ref{fig:a_red_green_fired}, observamos que a Rede de Petri conseguiu representar eficientemente a mudança de estado do semáforo A para aberto, enquanto os demais permaneciam fechados.
A habilitação seletiva das transições demonstra a capacidade da Rede de Petri de modelar sistemas com estados distintos.

A subsequente restituição das fichas, representada na Figura~\ref{fig:a_yellow_red_fired}, oferece uma visão do processo de retorno ao estado inicial após o fechamento do semáforo A.
A rede restabeleceu o equilíbrio nos lugares vermelhos de todos os semáforos, novamente proporcionando a capacidade de transição para qualquer estado.

Em síntese, os resultados deste estudo destacam a eficácia das Redes Coloridas de Petri na modelagem dinâmica de sistemas como cruzamentos de semáforos.
A capacidade de representar estados, transições e interações complexas entre componentes faz das Redes de Petri uma ferramenta valiosa para análise e simulação de sistemas de controle de tráfego, evidenciando seu poder de modelagem.
Essa abordagem oferece uma base sólida para futuras investigações e aplicações práticas na otimização de sistemas de sinalização de tráfego.