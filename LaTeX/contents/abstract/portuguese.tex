O artigo propõe uma abordagem simplificada para a modelagem de um cruzamento viário, focando exclusivamente na interação entre os quatro semáforos presentes.
Utilizando Redes Coloridas de Petri como estrutura de modelagem, o estudo busca oferecer uma representação clara da dinâmica de controle de tráfego no cruzamento.
Ao se concentrar na comunicação entre os semáforos, o artigo busca simplificar o entendimento do sistema, destacando os aspectos essenciais para uma compreensão robusta do fluxo de veículos no cruzamento.
Essa abordagem direcionada visa fornecer uma base sólida para futuras análises e otimizações, contribuindo para o desenvolvimento de sistemas de controle de tráfego mais eficientes e adaptáveis.